\documentclass[a4paper,12pt]{article}
\title{Robust Real-Time Face Recognition}
\author{B.Lagerwall & S.Viriri \\ Syazman Irman 1112700321}
\begin{document}
\maketitle
\tableofcontents
\newpage
\section{Summary of Research Proposal}
\hspace{1cm}Face detection and Face recognition has their on ways of recognition and detection. In this research paper describe and discuss the mathematical ways require to form face detection and face recognition at the exact moment needed.The moment we want to recognise or identified a person is sim- ply by looking at their faces or trajectory image. It is just as simple as remembering faces than name that are usually complicated or long. There are two stages in identifying any potential faces within image. The first step is face detection which involve recognising any faces image. The second stage is face recognition which involve taking any past or existing data and compared with faces images.

\hspace{1cm}The expected output from this research paper is to find a solution in face recognition and face detection to be very precise and accurated in real- time as it will help in many expect and  department.To assess the complete system, face detection and face recog- nition must meet the requirement needed and be analysed for high quality result. For testing purpose the subsystem were built in differ- ent type of modules that can be tested individually.

\hspace{1cm}This paper can be conclude that at various algorithm for face detection and face recognition it can be combined for experience a real time result with high quality output.Furthermore the it has been shown that the algorithm is robust since detection and recognition rates are not affected by the size of images. Thus this research should be expand for further method to revolutionary the face recognition system in the future.


\section{Introduction}
\hspace{1cm}Everyday actions are increasely being handled electronically, instead of pencil and paper or face to face. this growth is electronic transaction results in demand for fst and accurate user identification and authentication. Access codes for building, banks account and computer system often use PINs for identification and security clearances. Face recognition technology may solve this problem since a face is undeniably connected to its owner expect in case of identical twins.

\newpage
\section{Justification on Research}
\hspace{1cm}A biometric is a unique, measurable characteristic of human being that can be used to automatically recognized an individual or verify and individual's identity. Biometric can measure both physiological and behavioural characteristic.
For physiological biometrics is based on measurements and data derived from direct measurement of part of the human body. Behavioural biometrics is based on measurement and data derived from action.


\section{Research Objectives}
\hspace{1cm}Objective of this research paper is stated below :-
\begin{itemize}
\item Verification OF system
\item Identification database
\item Capture Images
\item Extraction Image
\item Comparison Images
\item Match/ Non-Match Face images
\end{itemize}


\section{Literature Review}
\hspace{1cm}Facial recognition software is based on the ability to first recognize faces, which is a technological feat in itself. If you look at the mirror, you can see that your face has certain distinguishable landmarks. These are the peaks and valleys that make up the different facial features. Here are few nodal points that are measured by the software.
\begin{itemize}
\item 1. distance between the eyes
\item 2. width of the nose
\item 3. depth of the eye socket
\item 4. cheekbones
\item 5. jaw line
\item 6. chin
\end{itemize}



\section{Research Methodology}
\hspace{1cm} Software Detection- when the system is attached to a video surveilance system, the recognition software searches the field of view of a video camera for faces. If there is a face in the view, it is detected within a fraction of a second. A multi-scale algorithm is used to search for faces in low resolution. The system switches to a high- resolution search only after a head-like shape is detected. Software Alignment- Once a face is detected, the system determines the heads position, size and pose. A face needs to be turned at least 35 degrees toward the camera for the system to register it. 

\hspace{1cm}Software Normalization-The image of the head is scaled and rotated so that it can be registered and mapped into an appropriate size and pose. Normalization is performed regardless of the heads location and distance from the camera. Light does not impact the normalization process. Software Representation-The system translates the facial data into a unique code. This coding process allows for easier comparison of the newly acquired facial data to stored facial data. Software Matching- The newly acquired facial data is compared to the stored data and (ideally) linked to at least one stored facial representation.

\hspace{1cm}The system maps the face and creates a faceprint, a unique numerical code for that face. Once the system has stored a faceprint, it can compare it to the thousands or millions of faceprints stored in a database. Each faceprint is stored as an 84-byte file.

\hspace{1cm} Factors such as environmental changes and mild changes in appearance impact the technology to a greater degree than many expect.For implementations where the biometric system must verify and identify users reliably over time, facial scan can be a very difficult, but not impossible, technology to implement successfully

\newpage
\section{Reference / Bibliography}
\begin{itemize}
\item www.biometricgroup.com/wiley
\item www.biometrics.gov
\item M. D. Abramo!, P. J. Magelhaes, and S. J. Ram. Image processing with imagej. Biophotonics International, 11(7):36–42, 2004.

\item G. Bradski. The OpenCV Library. Dr. Dobb’s Journal of Software Tools, 2000.

\item  W. Hoschek. The Colt Distribution: Open Source Libraries for High Performance Scientific and Technical Computing in Java. CERN, Geneva, 2004.

\item  H. Moon and P. J. Phillips. Computational and performance aspects of PCA-based face-recognition algorithms. 30(3):303–321, 2001.

\item  P. Phillips, H. Moon, S. Rizvi, and P. Rauss. The feret evaluation methodology for face recognition algorithms. IEEE Trans. Pattern Analysis and Machine Intelligence, 22:1090–1104, 2000.

\item  P. Phillips, H. Wechsler, J. Huang, and P. Rauss.The feret database and evaluation procedure for face recognition algorithms. Image and Vision Computing J, 16(5):295–306, 1998.

\item  M. Rahman, J. Ren, and N. Kehtarnavaz. Real-time implementation of robust face detection on mobile platforms. In Proceedings of the 2009 IEEE International Conference on Acoustics, Speech and Signal Processing, ICASSP ’09, pages 1353–1356,Washington, DC, USA, 2009. IEEE Computer Society.
\item H. A. Rowley. Neural network-based face detection. PhD thesis, Pittsburgh, PA, USA, 1999. AAI9950035.

\item  H. A. Rowley, S. Baluja, and T. Kanade. Human Face Detection in Visual Scenes. In Advances in Neural Information Processing Systems 8, pages 875–881, 1995. 
\item F. S. Samaria and A. C. Harter. Parameterisation of a stochastic model for human face identification. In Workshop on Applications of Computer Vision,1994.

\item M. A. Turk and A. P. Pentland. Face recognition using eigenfaces. In Proceedings. 1991 IEEE Computer Society Conference on Computer Vision and Pattern Recognition, pages 586–591. IEEE Comput. Sco. Press, 1991.

\item  P. Viola and M. J. Jones. Robust real-time face detection. Int. J. Comput. Vision, 57:137–154, May 2004.
\end{itemize}

\end{document}
